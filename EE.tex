\documentclass[12pt]{article}
\usepackage[utf8]{inputenc}
\usepackage{float}
\usepackage{amsfonts}
\usepackage{amssymb}
\usepackage{amsmath}
\usepackage{amsthm}
\usepackage{array}
\usepackage{graphicx}
\usepackage[margin=1in]{geometry}


\graphicspath{ {media/} } 

\setlength{\parskip}{1em}
\setlength{\parindent}{0em}
\fontsize{12pt}

\newtheorem{theorem}{Theorem}
\newtheorem*{definition}{Definition}

\title{Extended Essay}
\author{Michalis Pardalos}

\begin{document}

\maketitle

\tableofcontents

\section{Introduction}

\section{Divisibility}

    \subsection{Modular Arithmetic}
        An understanding of linear congruences and modular arithmetic is necessarry for the following proofs.

        Linear congruences have to do with division remainders. When two integers $a, b$ are congruent modulo
        a third integer $m$ they leave the same remainder when divided by $m$. Formally:
        %
        \begin{equation}\label{eq:congr_def_long}
            \begin{gathered}
                a \equiv b (mod\: m) \\
                \iff \exists\ l,m,r \in \mathbb{Z}: a = qm + r\ and\ b = lm + r
            \end{gathered}
        \end{equation}
        %
        An alternate definition is that 
        \begin{equation*}
            a \equiv b (mod\: m) \iff m | a-b
        \end{equation*}
        These two definitions are equivalent
        \begin{proof}
            Starting from (\ref{eq:congr_def_long})
            \begin{equation*}
                \left. 
                    \begin{aligned}
                        a = qm + r\\ 
                        b = lm + r
                    \end{aligned}
                \right\}
                (a - b) = (q-l)\times m :\: (q-l) \in \mathbb{Z}\quad \iff m | (a-b)
            \end{equation*}
        \end{proof}

        When performing arithmetic inside a modulo, addition, subtraction, and multiplication behave
        exactly as one would expect. Namely, given $a\equiv b\pmod{m}$ and $c\equiv d\pmod{m}$:
        \begin{align*}
            a + c &\equiv b + d \pmod{m}\\
            a - c &\equiv b - d \pmod{m}\\
            ka 	  &\equiv kb \pmod{m}\ \forall k \in \mathbb{Z} 
        \end{align*}

        Division, however, must be treated differently:
        \begin{gather*}
            \text{given } d = gcd(k, m)\\
            kx \equiv ky \pmod{m} \implies x \equiv y \pmod{\frac{m}{d}}
        \end{gather*}
        Note, that if $gcd(k, m) = 1$ 
        \begin{equation*}
            kx \equiv ky \pmod{m} \implies x \equiv y \pmod{m}
        \end{equation*}


\section{Primality tests}
    \subsection{Trial division}

    This is perhaps the most intuitive of all the primality tests. It consists of simply of 
    checking whether a number $p$ (the one being tested as prime) is divisible by all primes
    between 1 and $p$. This test however is grossly inefficient, due to the number of trial 
    divisions that have to be performed for larger integers. There exists however one crucial 
    theorem that provides an optimisation to this test:
    \begin{theorem}
        All composites have a prime factor less than or equal to their square root. Formally:\\
        $$\forall \text{ composite } n \: \exists \text{ prime } p:p|n \: \& \: p \leq \sqrt{n}$$
    \end{theorem}
    %
    This means that is enough to check all primes less than $\sqrt{n}$ for whether they divide $n$, 
    to ensure that $n$ is prime. Here is a proof by contradiction to this theorem:
    %
    \begin{proof}
    Since p is composite, $p=ab$ for some integers $1<a<n$, $1<b<n$.\\
    Suppose $a>\sqrt{n}$ and $b>\sqrt{n}$. Then, $ab>n \Rightarrow n>n$. Contradiction.\\
    Therefore, $a \leq \sqrt{n}$ or $a \leq \sqrt{n}$. \\
    Suppose (without loss of generality) that $a \leq \sqrt{n}$.\\
    There exists prime $p: p|a$. Note, that if a is prime, $p = a$, but this does not affect the proof\\
    $p|a\:\&\:a|n \Rightarrow p|n$. But $p|a \Rightarrow p \leq a \leq \sqrt{n}$\\
    \end{proof}
    %
    Even with this optimization, trial division is still slow for very large numbers 

    \subsection{Sieve of Eratosthenes}

    This method allows for very quick computation of all primes up to a certain limit. Here is how it works:
    \begin{enumerate}
        \item Write down all numbers up to the limit you have set.

            \includegraphics[scale=0.6]{base}
        \item Starting with 2, mark all its multiples, and note 2 as a prime

            \includegraphics[scale=0.6]{2}\\
        \item Repeat the same process, skipping any marked numbers, and marking as prime each unmarked 
            number you encounter

            \includegraphics[scale=0.6]{3}\\
            \includegraphics[scale=0.6]{5}\\
            \includegraphics[scale=0.6]{7}\\

    \end{enumerate}

    This method is extremely effective in finding prime numbers, and can even be performed by hand for 
    moderately small numbers. However, when applied to very large large limits, the method quickly presents 
    a problem of the memory required. In order for the sieve to work all found composites must be stored, so 
    that they can be skipped. The memory space required for the algorithm increases with the limit set, 
    making it impractical for computing very large primes.

    \subsection{Fermat's  Little Theorem}
    Fermat's theorem (not to be confused with Fermat's \emph{last} theorem, its better known big brother)
    states that:
    \begin{theorem}[Fermat's Little Theorem]
        $p$ is prime and $p \not|\ a$ $\implies a^{p-1} \equiv 1 \pmod{p}$
    \end{theorem}

    \begin{proof}
        Consider the numbers 0, 1a, 2a, ..., (p-1)a\\
        Suppose two of them $ka,\ la: 1\leq k < l \leq (p-1)$ are congruent modulo $p$, i.e. 
    %
        \begin{align*}
                  ka &\equiv la &\pmod{p}&\\
            \iff  k  &\equiv l  &\pmod{p}&\ \text{since p prime and $p \not| a$}
        \end{align*}
    %
        But the integers in $[0, (p-1)]$ form a complete residue system modulo $p$, and  $1 \leq k<l \leq (p-1)$, 
        meaning that $k \not\equiv l \pmod p$. \emph{Contradiction}. Therefore, 
    %
        \begin{equation*}
            ka \not\equiv la \pmod{p}\qquad \forall\ k,l \in [0, (p-1)], k \not= k
        \end{equation*}    
    %
        So, $0...(p-1)a$ are all incongruent to each other and so must be, in some order, congruent to the least
        residue system modulo $p$, i.e. $0...(p-1)$
    %
        \begin{align*}
            a(2a)...(p-1)a &\equiv 1\times 2\times 3...\times (p-1) &\pmod{p}\\
            a^{p-1}(p-1)!  &\equiv (p-1)!                           &\pmod{p}\\
            a^{p-1}        &\equiv 1                                &\pmod{p}
        \end{align*}
    \end{proof}
            
    \subsection{Fermat test}
    Based on Fermat's theorem, a primality test can be devised.

    Given an even integer $n$ that we want to test as prime we can check whether $a^{n-1} \equiv 1 \pmod{n}$ 
    holds for a number of integers $1<a<n$. This test cannot, by itself, prove the primality of a given 
    number. It can, however, prove compositeness: if $a^{n-1} \not\equiv 1 \pmod{n}$ then m is necessarily 
    composite since Fermat's theorem holds for all primes.

\section{RSA encryption}
    RSA is currently, one of the most popular encryption systems in the world. It belongs to a 
    category of encryption systems known as public-private key. This means, that it allows two
    users to exchange information securely, without having to exchange a shared encryption key 
    beforehand, as it would be required by classical encryption systems such as Caesar or 
    substitution ciphers. 

    The system is based on the theory of congruences, and, specifically, on Euler's Theorem.

    \subsection{The Euler $\phi$-function}
    This function is necessary for Euler's theorem and, subsequently, RSA encryption. The definition of the
    function is quite simple.
    \begin{definition}
        $\phi(n)$ is defined as the \emph{number} of integers less than or equal to $n$ that are relatively prime
        to $n$. By convention, $\phi(1) = 1$.
    \end{definition}
    So, for example, $\phi(10) = 4$, $\phi(25) = 20$.

    A quick conclusion that can be drawn is that, for prime $n$, $\phi(n) = n - 1$, since all integers less than
    or equal to $n$, aside from $1$ and $n$ itself, are relatively prime to $n$.

    \subsection{Euler's Theorem}
    \subsection{Definitions \& Terminology}
    Given here are some terms and definitions that will be used:
    %
    \begin{table}[H]
        \begin{tabular}{ | m{5em} | p{30em} | }
            \hline
            Term      & Definition\\
            \hline
            plaintext & The message that will be encrypted, in \emph{plain text}, i.e. before it is
                        encrypted. Usually represented as $M$\\
            \hline
            ciphertext & The message after it has been encrypted. This is what is sent from one
                         user to the other. Without knowledge of the public and private key it 
                         is impossible to recover the plaintext using just the ciphertext. Usually
                         represented as $C$\\
            \hline
        \end{tabular}
    \end{table}
    %
    \subsection{Encryption}
    RSA requires a number of values that will be used for the encryption/decryption, known as keys.
    The first of these are two primes $p,q$. These are usually very large numbers, in the order
    of a few hundred or even thousand digits, but for these examples we will use much smaller primes,
    just for the sake of simplicity. Next we assign $n=p \times q$. Finally, we need an odd 
    integer $k$ such that $gcd(k, \phi (n)) = 1$ 

    The values we will use for the example are: $p=13$, $q=11$, $n=143 \implies \phi (n) = 120$
    and $k = 23$

    Let's begin with the encryption of a plaintext message. Firstly, the message must be encoded
    as a number. The simplest way of doing that would be by converting character into a number 
    and encrypting each character separately. In real world applications, due to the complicated nature of the 
    arbitrary messages that have to be transmitted, this encoding cannot be used, but it will suffice for this 
    example.

    So, for example, the message "HELLO" would become $08\ 05\ 12\ 12\ 15$, using two digits for 
    each character.

    Next, comes the actual encryption. With $C$ as the ciphertext and $M$ as the plaintext:
    \begin{equation*}
        C \equiv M^{k} \pmod{n}
    \end{equation*}

    Using the keys we defined, the ciphertext becomes $31\ 18\ 27\ 27\ 24$, which can then be safely
    transmitted, so that it can be decrypted.

    \subsection{Decryption}
    Having received the ciphertext, we can recover the original message using the congruence
    \begin{align*}
        M \equiv C^j \pmod{n})\\
        \text{where } kj \equiv 1 \pmod{\phi (n)}
    \end{align*}

    Applying this to our example, we first need to solve $kj \equiv 1 \pmod{\phi (n)}$ for j

    \begin{proof}
        Start with the encryption congruence: 
        \begin{align*}
            C & \equiv M^{k} \pmod{n}\\
            M & \equiv C^{\frac{1}{k}} \pmod{n}
        \end{align*}
        We need, therefore, to find $\frac{1}{k} \pmod{n}$, i.e. the \emph{multiplicative inverse}
        of $k$ in modulo $n$. By assumption, 
        \begin{align*}
            gcd(k, \phi (n)) = 1 \\
            \implies \exists\ x: kx \equiv 1 \pmod{\phi (n)}
        \end{align*}
        Let $j$ be a solution to the congruence in the least residue system modulo $\phi (n)$ such that 
        $j \in \left[1, \phi (n) \right)$. Therefore, $\exists z: kj = z\phi (n) + 1$

        Going back to the encryption congruence we can raise it to the $j^{th}$ power to get
        \begin{align*}
            C^j & \equiv M^{kj}\\ 
                & \equiv M^{z\phi (n) + 1}\\
                & \equiv M\times M^{z \phi(n)}\\
                & \equiv M\times (M^{\phi (n)})^z\\
                & \equiv M\times 1^z                &\text{(by Euler's theorem)}\\
                & \equiv M \pmod{n}
        \end{align*}
    \end{proof}

\end{document}

